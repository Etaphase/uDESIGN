\documentclass[a4paper,10pt]{article}
\usepackage[utf8]{inputenc}
\usepackage{fullpage}
\usepackage{graphicx}
\usepackage{hhline}


\title{A Small Theorem on the Interval Approach Criterion for Systems of Linear Equations.}
\author{Isaac T Yonemoto}

\begin{document}

\maketitle

\section{Motivation}

The development of Unums encourages reformulation of computational mathematics to approach
numerical problems, keeping in mind traditional performance metrics, such as speed and energy
consumption, while still yielding mathematically accurate solutions.  Traditional numerical
systems, such as IEEE floating points, abandon accuracy for performance.  Symbolic systems
incur high calculation costs and still may encounter situations without closed form solutions.
Interval arithmetic can have exploding uncertainty under certain operations without clear
guidance on how to resolve this.  Unums, however, provide a natural solution to the exploding
uncertainty, by way of methods such as the Ubox method.

For solving systems of linear equations, the naive solution is to first perform a gauss-jordan
elimination and then whittle down the solution set using Uboxes.  This is inefficient and cannot
distinguish between a unit cell that contains a solution and a unit cell that merely happens to
be at the intersection of all of the hyperplanes.  Moreover, the search time proceeds with time
O(something large)

\section{The Interval Approach (uSlice) method}

To address this issue, I developed a strategy called the interval approach method.  The procedure
is as follows:  Select a dimension.  Partition a superset of the solved interval in this dimension 
with equal width (simplest to choose a power of two), called ``uslices''.  Apply the uslices as
values to the system of linear equations and measure the maximum error from the constant vector of the
linear equations.  Because the equations are linear, conceptually the ``outside'' slices should 
have greater error than the ``inside'' slices, with a few exceptions that are otherwise easily 
solved (e.g. an irrelevant dimension that trivially reduces to a smaller system of linear equations).

In practice, however, this interval approach method fails in the event that a bounding interval
is \emph{asymmetric} about the solution.  This document describes the derivation of an ``Interval
Approach Criterion'' which guarantees that the uslices with lowest error bound the solution in
the exact case and must contain the solution in the inexact case.

\section{Derivation of the Interval Approach Criterion: 2-dimensions, positive matrix, exact solution}

Given a matrix $ M = \left( \begin{array}{cc} A & B \\ C & D \end{array} \right)$ with constant
vector $v = \left( \begin{array}{c} v_1 \\ v_2 \end{array} \right)$ or written as            
$\left( \begin{array}{cc|c} A & B & v_1 \\ C & D & v_2 \end{array} \right)$.  The solution
$s = \left( \begin{array}{c} x \\ y \end{array} \right)$ and without loss of generality 
over variable order y is bounded by an asymmetric open interval $y \in (y - \delta_l, y + \delta_h)$,
and we examine the following intervals for x: $v_{far} = (x - 2\epsilon, x - \epsilon)$ and 
$v_{near} = (x - \epsilon, x)$

The first equation evaluates as follows:

$(M \cdot v_{far})_1 = A(x - 2\epsilon, x - \epsilon) + B(y - \delta_l, y + \delta_h)$

And:

$(M \cdot v_{near})_1 = A(x - \epsilon, x) + B(y - \delta_l, y + \delta_h)$

Subtracting $v_1 = Ax + By$ from these intervals results in the following error intervals:

$(-2\epsilon A - B\delta_l, -\epsilon A + B\delta_h)$

And:

$(-\epsilon A - B\delta_l, B\delta_h)$


Since the outer values of the intervals are necessarily extrema, the ``maximal'' error for these
uSlices is attained at the absolute value.  The assertion of interval approach is thus satisfied 
when $max(2\epsilon A + B\delta_l, |-\epsilon A + B\delta_h|) > max(\epsilon A + B\delta_l, B\delta_h)$.
Since always $\epsilon A + B\delta_l < 2\epsilon A + B\delta_l$, this value is not in contention.
We must only check that $B\delta_h$ is not greater than either critical value.

\[\epsilon > \frac{(B\delta_h - B\delta_l)}{2A}\] satisfies this condition when $2\epsilon A + B\delta_l > |-\epsilon A + B\delta_h|$


\end{document}
